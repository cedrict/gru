\documentclass[a4paper]{article}
\usepackage[cm]{fullpage}
\usepackage{url}
\usepackage{listings}
\usepackage{upgreek}
\usepackage{siunitx}
\usepackage{bm}
\usepackage{amsmath}
\usepackage{amsfonts}

\newcommand{\K}{{\mathbb{K}}}
\newcommand{\G}{{\mathbb{G}}}


\usepackage[maxnames=6]{biblatex}
\addbibresource{bibliography.bib}



\title{GRU}
\author{C. Thieulot \& F. Gueydan}

\begin{document}
\maketitle



%%%%%%%%%%%%%%%%%%%%%%%%%%%%%%%%%%%%%%%%%%%%%%%%%%
\section{Mass and momentum conservation equations}


\begin{eqnarray}
\vec\nabla \cdot \bm \sigma + \rho \vec{g} &=& \vec{0} \\
\vec\nabla \cdot \vec\upnu &=& 0
\end{eqnarray}

The full stress tensor is given by
\[
\bm\sigma = -p {\bm 1} +  \bm \tau
\]
where ${\bm 1}$ is the unit matrix, $p$ is the pressure and 
${\bm\tau}$ is the deviatoric stress tensor which can be 
written as
\[
\bm\tau = 2 \eta \dot{\bm \varepsilon}(\vec\upnu)
\]
where $\eta$ is the viscosity, $\vec{\upnu}=(u,v)$ is the velocity vector, 
and $\dot{\bm \varepsilon}(\vec\upnu)$ is the (deviatoric) 
strain rate tensor.

Putting it all together we obtain:
\begin{eqnarray}
-\vec\nabla p + \vec\nabla \cdot (2 \eta \dot{\bm \varepsilon}(\vec\upnu)) + \rho \vec{g} &=& \vec{0} \\
\vec\nabla \cdot \vec\upnu &=& 0
\end{eqnarray}

In what follows we assume the buoyancy forces are negligible, i.e. 
the term $\rho \vec{g}$ is neglected.

For the time being we assume the system is isothermal 
so the energy equation is not solved.


%%%%%%%%%%%%%%%%%%%%%%%%%%%%%%%%%%%%%%%%%%%%%%%%%%
\section{Numerical methods}

The mass and momentum conservation equations are 
solved by means of the FE method. 
$Q_2\times Q_1$ elements are used \cite{thba22}.
The domain is a rectangle of size $L_x \times L_y$
with the lower left corner at $(x,y)=(0,0)$.
The mesh is composed of \lstinline{nelx}$\times$\lstinline{nely} elements.
Boundary conditions are as follows: $\upnu_y=0$ is prescribed on 
all boundaries, while $\upnu_x=\pm \upnu_0$ is prescribed at the 
top and at the bottom (of opposite sign) so that the background 
strain rate is given by 
\[
\dot\varepsilon_b = \frac{\upnu_0}{L_y}
\]
Typically we set $L_y=1~\si{\cm}$ and 
$\dot\varepsilon_b =10^{-15}~\si{\per\second} $ so that 
$\upnu_0= \dot\varepsilon_b L_y \simeq 3.2\cdot 10^{-8}~\si{\cm\per year}$.
Calculations are performed until a strain value $\gamma=2$ 
has been reached, i.e. 
$t_{final} = \gamma/ \dot\varepsilon_b \simeq 2 \times 10^{15}~\si{\second} 
\simeq 64~\si{Myr}$.


\begin{center}
INSERT here fig of velocity is square
\end{center}

A cloud of passive markers (hereafter called swarm\footnote{\url{https://en.wikipedia.org/wiki/Swarm_behaviour}}) is 
place in the domain. 
The \lstinline{nmarker_per_dim} parameter controls the number of
markers placed in each element at the beginning of the simulation:
\lstinline{nmarker_per_element=nmarker_per_dim**2}
while the total number of markers in the domain is then 
\lstinline{nmarker=nel*nmarker_per_element}.
Markers are initially placed on a regular grid of positions inside each element, as shown in 
Fig.XX

\begin{center}
INSERT here figure with markers at t=0
\end{center}

Each marker carries a lot of information stored in the \lstinline{swarm_xxx}
arrays, where xxx stands for the name of the field, such as x,y,gs,eta,... 

At each time step markers are localised (i.e. we find in which element they reside
in, the (effective) strain rate is interpolated onto them and passed as 
argument to the \lstinline{viscosity} function which, based on the 
temperature $T$ and the grain size $d$ of each marker computes the 
effective viscosity (see Section~\ref{sec:visc}).

This viscosity is then averaged inside the element and is used to build
the elemental matrix $K_\eta$.

The FE matrix is assembled using \lstinline{lil_matrix},
converted to CSR format and then passed to a direct solver alongside the rhs vector.
Node that the code is based on the codes available in the educational Fieldstone project
and is therefore not optimised for performance.



Following a standard approach, the discretised Stokes equations yield
the following linear system
\[
\left(
\begin{array}{cc}
\K & \G \\
\G^T & 0 
\end{array}
\right)
\cdot
\left(
\begin{array}{c}
\vec{\cal V} \\ 
\vec{\cal P}
\end{array}
\right)
=
\left(
\begin{array}{c}
\vec{f} \\ 
\vec{h}
\end{array}
\right)
\]
where $\vec{\cal V}$ is the vector containing all velocity degrees of 
freedom (size { NfemV}={NV}*{ ndofV})
and $\vec{\cal P}$ is the vector containing all pressure degrees of freedom 
(size { NfemP}={ NP}*{ ndofP}).



After the FE matrix has been built, the linear system is solved, and the 
nodal velocity is used to advect the markers. 
For simplicity we resort to an Euler step, i.e.
\[
\vec{\text x}_i(t+\delta t) = \vec{\text x}_i(t) + \vec\upnu_i \; \delta t
\qquad
i=1,...nmarker 
\]
The timestep $\delta t$ is controled by a CFL condition with $C=0.25$:
\[
\delta t = C \frac{h}{\max |\vec\upnu|_\Omega}
\]
where $h$ is the element size and $C\in[0,1[$.



At the beginning each marker is assigned a grain size $d=1000~\si{\micro\meter}$.
In the middle of the domain random noise is added as shown in Fig.YYY
\begin{center}
INSERT fig of initial d
\end{center}
The grain size value carried by each marker is evolved according the 
equation presented in Section~\ref{seq:gsev}.

Because the effective viscosity of each marker (and therefore each element)
depends on the strain rate, we carry out simple Picard nonlinear iterations 
which stop when the 2-norm of the relative change of the velocity field between two consecutive 
iterations is less than a set tolerance of $10^{-4}$.


%%%%%%%%%%%%%%%%%%%%%%%%%%%%%%%%%%%
\section{Strain rate decomposition \label{sec:visc}}

%This stone is based on \textcite{gupr14} (2014). 
The strain rate is to be decomposed into its various contributions 
coming from the different deformation mechanisms, dislocation creep,
diffusion creep, disGBS and low-temperature plasticity:
\[
\dot\varepsilon = \dot\varepsilon_{dsl} + \dot\varepsilon_{dif} + 
\dot\varepsilon_{gbs} + \dot\varepsilon_{exp} 
\]
with
\begin{eqnarray}
\dot{\varepsilon}_{dsl}&=&A_{dsl}\exp\left(-\frac{Q_{dsl}}{RT} \right) \tau^{n_{dsl}}  \\
\dot{\varepsilon}_{dif}&=&A_{dif}\exp\left(-\frac{Q_{dif}}{RT} \right) \tau^{n_{dif}} d^{m_{dif}} \\
\dot{\varepsilon}_{gbs}&=&A_{gbs}\exp\left(-\frac{Q_{gbs}}{RT} \right) \tau^{n_{gbs}} d^{m_{gbs}} \\
\dot{\varepsilon}_{exp}&=&A_{exp}\exp\left[-\frac{Q_{exp}}{RT} \left(1 -\frac{\tau}{\tau_p}\right)^{n_{exp}} \right]   
\end{eqnarray}
where $d$ is the grain size, $m$ is the grain size exponent, $\tau_p$ is the Peierls stress defined
for low-temperature plasticity.


As explained in Section~\ref{MMM-ss:srpart}, there is one major problem with the equations above:
Assuming $\dot\varepsilon$ and temperature $T$ known (as well as all the material parameters $A$, $Q$, $n$, ...),
and that the deformation mechanisms are in series and subjected to the same deviatoric stress $\tau$,
we must find $\tau$ such that
\[
{\cal F}(\tau) = \dot\varepsilon -  \dot\varepsilon_{dsl}(\tau) 
-\dot\varepsilon_{dif}(\tau) -\dot\varepsilon_{gbs}(\tau) - \dot\varepsilon_{exp}(\tau) =0
\]
Unfortunately, this equation is non-linear in $\tau$ so that finding its zero(es) is not
straightforward. A Newton-Raphson\footnote{\url{https://en.wikipedia.org/wiki/Newton's_method}}
algorithm is then used. How to build such an algorithm is presented in Section~\ref{MMM-ss:srpart}
but we will here use an existing python function.
We load \lstinline{scipy.optimize} module and use the \lstinline{newton} function\footnote{\url{
https://docs.scipy.org/doc/scipy/reference/generated/scipy.optimize.newton.html}}
which finds a zero of a real or complex function using the Newton-Raphson (or secant or Halley’s) method.
Once $\tau$ (\lstinline{tau_NR}) has been found, it can then be inserted in the strain rate equations above and
the strain rate partitioning is then complete.

Note that the $\dot\varepsilon_e$ is the effective strain rate defined as 
\[
\dot\varepsilon_e = \sqrt{\frac12 (
\dot\varepsilon_{xx}^2+
\dot\varepsilon_{yy}^2)+
\dot\varepsilon_{xy}^2
)} 
\]

%%%%%%%%%%%%%%%%%%%%%%%%%%%%%%%%%%%
\section{Grain size evolution \label{seq:gsev}}

Following \textcite{prgu09}, we simulate a dynamic grain size reduction
by using the following grain size evolution law \textcite{brcp99}, that relates
the rate of change of grain size $\dot{d}$ to the 
deformation rate $\dot{\varepsilon}$ according to

\[
\dot{d} = -\frac{\dot\varepsilon}{\dot\varepsilon_T} (d-d_\infty)
\]
where $d_\infty$ obeys the following piezometric 
relationship
\[
d_\infty = B \tau^{-p}
\]
where $d_\infty$ is the recrystallized grain size defined
by the field boundary hypothesis

FG: Please clean this 



\newpage
\printbibliography
\end{document}
